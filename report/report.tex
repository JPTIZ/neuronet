%-------------------------------------------------------------------------------
\documentclass[twocolumn]{article}

%-------------------------------------------------------------------------------
% Packages
\usepackage[portuguese]{babel}
\usepackage{environ}
\usepackage[margin=2cm]{geometry}
\usepackage{graphicx}
\usepackage{hyperref}
\usepackage{minted}
\usepackage{xcolor}

%-------------------------------------------------------------------------------
% User-commands
\newcommand{\todo}[1]{{\color{red}{#1}}}

\NewEnviron{superframe}{%
    \begin{center}
        \fbox{\setlength{\fboxsep}{1em}\fbox{\parbox{5.5in}{%
            \BODY{}
        }}}
    \end{center}
}

\newmintedfile[textfile]{text}{autogobble, breaklines}

%-------------------------------------------------------------------------------
% Project configs
\title{Relatório de I.A.: Redes Neurais (Trabalho 5)}
\author{Cauê Baasch de Souza \\
        João Paulo Taylor Ienczak Zanette}
\date{\today}

%-------------------------------------------------------------------------------
\begin{document}
    \maketitle{}

    \section{Resumo do projeto}

    \begin{description}
        \item [Linguagem:] Python 3.7
        \item [Biblioteca de Redes Neurais utilizada:] sklearn~\cite{sklearn}.
    \end{description}

    \section{Configuração dos experimentos}

    Os experimentos foram realizados tomando como base dois conjuntos de dados
    já disponibilizados pelo professor na plataforma Moodle, sendo um para
    treinamento da rede neural e outro para testes. Ambos os conjuntos são
    formados por tuplas no formato $(output, pixel_1, pixel_2, \ldots,
    pixel_n)$, em que $output$ é um número simbolizando a categoria esperada
    para a análise do conjunto de pixeis denotados por $pixel_{i}$.

    O tratamento da rede neural foi separado em duas etapas: uma de
    treinamento, enviando à rede todas as tuplas do conjunto de treinamento em
    um grande lote, e outra para testes enviando as tuplas do conjunto de
    testes e, para cada teste, validando se a previsão da rede foi feita
    corretamente ou não.

    \section{Normalização dos Dados}

    A gente normalizou os dados.

    \section{Separação dos conjuntos de treinamento e teste}

    A gente separou os conjuntos de treinamento e teste.

    \section{Arquitetura de rede}

    A gente usou uma arquitetura de rede.

    \section{Quantos e quais experimentos foram feitos até chegar no resultado final}

    Vários (mentira, foram só uns 2 depois que funcionou com a lib).

    \section{Como foi o treinamento}

    Foi bem divertido, obrigado.

    \section{Qual a taxa de acertos da rede}

    66\% eu acho, por aí.

    \section{Matriz de confusão}

    Que.

    \section{Exemplos de objetos que foram mal-classificados pela rede}

    *Sigh* isso vai dar um trabalho\ldots

    \section{Fatos que você achou interessante}

    Implementar backtracking é um belo cu mas foi legal.

    \bibliographystyle{unsrt}
    \bibliography{refs}
    \nocite{*}
\end{document}
